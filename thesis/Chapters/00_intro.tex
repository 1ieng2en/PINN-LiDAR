\chapter{Introduction to Fourier-Enhanced DeepONet for Acoustic Signal Processing}

In the recent advancements of machine learning technologies, the combination of neural operators with mathematical transformations has shown immense promise in enhancing computational efficiency and predictive accuracy. This chapter introduces a novel approach that integrates Deep Operator Networks (DeepONets) with the Fast Fourier Transform (FFT), focusing specifically on applications in acoustic signal processing. We refer to this approach as \textit{Fourier-Enhanced DeepONet}.

DeepONets, originally designed by Lu et al., are powerful tools for learning mappings between infinite-dimensional spaces. When training DeepONets, a common challenge is the computational expense due to the large data dimensionality. To address this, we propose leveraging the FFT to transform data into the spatial Fourier domain, which allows for more efficient processing.

The core idea is to employ FFT to project input signals into frequency space, where the DeepONet can learn the mapping with potentially fewer resources. Let \( \mathbf{x}(t) \) be the continuous-time input signal. Applying the FFT provides the frequency domain representation:

\[
\mathbf{X}(f) = \int_{-\infty}^{\infty} \mathbf{x}(t) e^{-i 2 \pi f t} \, dt
\]

In this setup, the DeepONet is trained on the transformed data \( \mathbf{X}(f) \), effectively learning the operator in the frequency domain. This Fourier-enhanced training process can significantly improve both the speed and the accuracy of the predictive model, particularly in the context of Sound Field Reproduction (SFR).

\section{Applications in Acoustic Signal Processing}

Sound Field Reproduction and acoustic signal processing greatly benefit from precise field predictions and efficient sampling techniques. Traditional methods often require extensive computational resources and high-dimensional datasets. By applying the Fourier-Enhanced DeepONet approach, we aim to optimize the sampling strategy and improve SFR outcomes.

We conduct predictions and benchmark them against conventional methods such as direct time-domain simulations and standard neural network architectures. The hypothesis is that Fourier domain learning will yield superior performance by reducing the dimensionality of the problem and by taking advantage of the periodic properties of acoustic waves.

\section{Analysis and Discussion}

The examination of this method focuses on several key aspects:
1. \textbf{Efficiency}: Analyze computational costs and training times compared to traditional methods.
2. \textbf{Accuracy}: Evaluate predictive accuracy and error margins in SFR tasks.
3. \textbf{Robustness}: Assess the network's ability to generalize across different acoustic scenarios.

In summary, the Fourier-Enhanced DeepONet presents a promising avenue for advancing acoustic signal processing techniques, offering a robust framework for more efficient and accurate sound field predictions. The following chapters will delve into the technical details, experimental setups, and comprehensive evaluations of the method.